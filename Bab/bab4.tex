\chapter{Perancangan}
Bab ini akan membahas mengenai perancangan aplikasi. Perancangan aplikasi akan meliputi tampilan antarmuka, diagram kelas lengkap beserta dengan deskripsi dan fungsinya.

\section{Perancangan Antarmuka}

Agar pengguna dapat menggunakan perangkat lunak ini dengan nyaman, maka dibutuhkan antarmuka. Rancangan antarmuka yang dibuat terdiri dari satu \textit{frame} dan di dalamnya terdapat beberapa \textit{tab}.

\subsection{\textit{Tab Embed}}

\begin{figure}[H]
	\centering
	\includegraphics[scale=1.8]{Gambar/tab-embed}
	\caption{Tampilan \textit{tab Embed}} 
	\label{fig:1-tab-embed}
\end{figure}

Pada \textit{tab Embed} (dapat dilihat pada Gambar \ref{fig:1-tab-embed}) terdapat beberapa elemen sebagai berikut.

\begin{enumerate}
	\item \textbf{\textit{Text box secret message}}, pengguna dapat mengetikkan pesan rahasia yang akan disembunyikan pada bagian ini.
	\item \textbf{Tombol \textit{Embed}}, pengguna dapat menekan tombol ini untuk melakukan proses \textit{embedding}.
	\item \textbf{\textit{Text area stego-object}}, hasil proses \textit{embedding} yang merupakan \textit{stego-object} akan muncul pada bagian ini.
\end{enumerate}

\subsection{\textit{Tab Extract}}

\begin{figure}[H]
	\centering
	\includegraphics[scale=1.8]{Gambar/tab-extract}
	\caption{Tampilan \textit{tab Extract}} 
	\label{fig:2-tab-extract}
\end{figure}

Pada \textit{tab Extract} (dapat dilihat pada Gambar \ref{fig:2-tab-extract}) terdapat beberapa elemen sebagai berikut.

\begin{enumerate}
	\item \textbf{\textit{Text area stego-object}}, pengguna dapat menyalin \textit{stego-object} yang diterima pada bagian ini.
	\item \textbf{Tombol Ekstrak}, pengguna dapat menekan tombol ini untuk melakukan proses \textit{extracting}.
	\item \textbf{\textit{Text box secret message}}, hasil \textit{extracting} akan ditampilkan pada bagian ini.
\end{enumerate} 

\subsection{\textit{Tab} Tambah \textit{Cover}}

\begin{figure}[H]
	\centering
	\includegraphics[scale=1.8]{Gambar/tab-tambah-cover}
	\caption{Tampilan \textit{tab} Tambah \textit{Cover}} 
	\label{fig:3-tab-tambah-cover}
\end{figure}

Pada \textit{tab} Tambah \textit{Cover} (dapat dilihat pada Gambar \ref{fig:3-tab-tambah-cover}) terdapat beberapa elemen sebagai berikut.

\begin{enumerate}
	\item \textbf{\textit{Text box} judul}, pengguna dapat mengetikkan judul dari \textit{stego-cover} yang akan ditambahkan pada bagian ini.
	\item \textbf{Tombol Tambah \textit{Cover}}, pengguna dapat menekan tombol ini untuk menambahkan \textit{stego-cover} yang telah disalin pada no 3.
	\item \textbf{\textit{Text area stego-cover}}, pengguna dapat menyalin \textit{stego-cover} yang memenuhi persyaratan di bagian ini.	
\end{enumerate} 

\subsection{\textit{Tab} Tambah Sinonim}

\begin{figure}[H]
	\centering
	\includegraphics[scale=1.8]{Gambar/tab-tambah-sinonim}
	\caption{Tampilan \textit{tab} Tambah Sinonim} 
	\label{fig:4-tab-tambah-sinonim}
\end{figure}

Pada \textit{tab} Tambah Sinonim (dapat dilihat pada Gambar \ref{fig:4-tab-tambah-sinonim}) terdapat beberapa elemen sebagai berikut.

\begin{enumerate}
	\item \textbf{\textit{Text box} kata pada \textit{cover}}, kata pada \textit{cover} berarti kata yang ada pada \textit{stego-cover}, pengguna dapat mengetikkannya pada bagian ini.
	\item \textbf{\textit{Text box} sinonim}, pengguna dapat mengetik sinonimnya pada bagian ini.
	\item \textbf{Tombol Tambahkan}, pengguna dapat menekan tombol ini untuk memasukkan pasangan kata dan sinonim yang baru ke \textit{file database}.
	\item \textbf{\textit{Text Area} Error Log}, akan menampilkan daftar kata yang belum memiliki sinonim jika pengguna menekan tombol Periksa semua sinonim.
	\item \textbf{Tombol Periksa semua sinonim}, akan memeriksa semua kata dari tiap \textit{stego-cover} yang terdaftar dan menampilkan daftar kata yang belum memiliki sinonim.
\end{enumerate} 

\subsection{\textit{Tab} Petunjuk}

\begin{figure}[H]
	\centering
	\includegraphics[scale=1.8]{Gambar/tab-petunjuk}
	\caption{Tampilan \textit{tab} Petunjuk} 
	\label{fig:5-tab-petunjuk}
\end{figure}

\textit{Tab} Petunjuk (dapat dilihat pada Gambar \ref{fig:5-tab-petunjuk}) sebenarnya hanya dibuat untuk membantu pengguna yang baru pertama kali memakai perangkat lunak ini. Pada \textit{tab} ini berisi petunjuk-petunjuk yang dapat membantu pengguna untuk mengoperasikan perangkat lunak ini.

\section{Diagram Kelas Rinci}

Diagram kelas sebelumnya yang dapat dilihat pada Gambar \ref{fig:3_classdiagram} mengalami beberapa perubahan dan penambahan kelas.

\section{Analisis Diagram Aktivitas}

Diagram aktivitas dari perangkat lunak dibagi menjadi dua. Diagram aktivitas saat menyisipkan pesan dapat dilihat pada \ref{fig:4_activity-penyisipan}

\begin{figure}[H]
	\centering
	\includegraphics[scale=0.5]{Gambar/activity-penyisipan}
	\caption{Activity Diagram perangkat lunak steganografi saat menyisipkan} 
	\label{fig:4_activity-penyisipan}
\end{figure}

Dari diagram aktivitas dapat dijelaskan sebagai berikut:

\begin{enumerate}
	\item Pengguna memasukkan input berupa string (pesan rahasia)
	\item Pesan akan diubah ke bentuk ASCII
	\item Dokumen korpus akan dibuka satu per satu
	\item Untuk setiap dokumen yang dibuka akan dilakukan penyukuan kata dan dilakukan penyandian. Angka 1 untuk kata dengan jumlah suku kata genap dan angka 2 untuk kata dengan jumlah suku kata ganjil
	\item Pesan rahasia yang telah diubah ke bentuk ASCII dan dokumen yang telah disandikan akan dibandingkan. Dari sini akan dihasilkan dua kemungkinan:
	\begin{itemize}
		\item Jika ditemukan dokumen cocok dengan pesan rahasia, akan diambil kalimat yang sesuai
		\item Jika tidak ditemukan dokumen yang cocok, dari dokumen yang paling mirip akan dilakukan modifikasi (perubahan sinonim atau perubahan kalimat aktif/pasif)
	\end{itemize}
	\item Jika telah didapatkan kalimat yang sesuai, kalimat akan ditampilkan dan program selesai
\end{enumerate}

Diagram aktivitas ekstraksi dapat dilihat pada 

\begin{figure}[H]
	\centering
	\includegraphics[scale=0.5]{Gambar/activity-ekstraksi}
	\caption{Activity Diagram perangkat lunak steganografi saat proses ekstraksi} 
	\label{fig:4_activity-ekstraksi}
\end{figure}

Dari diagram aktivitas di atas dapat dijelaskan sebagai berikut:

\begin{enumerate}
	\item Pengguna akan memberikan input berupa stego-object dari program yang sama
	\item Sistem lalu akan menyukukan kata-kata yang ada. Setelah itu dilakukan penyandian, angka 0 untuk kata dengan jumlah suku kata genap dan angka 1 untuk jumlah ganjil
	\item Selesai melakukan penyandian, hasilnya akan diubah menjadi karakter ASCII yang sesuai
	\item Kini sudah didapatkan pesan dengan karakter ASCII, hasil akan ditampilkan pada layar
\end{enumerate}