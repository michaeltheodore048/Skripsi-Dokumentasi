0\chapter{Pendahuluan}
\label{chap:pendahuluan}

\section{Latar Belakang}
\label{sec:latarbelakang}

Tidak dapat dipungkiri bahwa teknologi membawa perubahan besar pada seluruh umat manusia di dunia. Dengan adanya internet yang merupakan salah satu produk dari kemajuan teknologi, jarak dan waktu bukanlah menjadi suatu masalah. Komunikasi antar benua, bahkan antar belahan dunia pun bukanlah masalah. Internet juga memudahkan pencarian segala informasi yang mulanya sulit ditemukan.

Namun segala sesuatu yang memiliki sisi positif, pasti ada sisi negatifnya. Informasi yang berserakan di internet dapat dilihat oleh siapa pun. Semua orang yang beraktivitas di internet dapat mengklaim dirinya sebagai siapa pun yang dikehendaki. Tidak terkecuali juga informasi pribadi yang seharusnya tidak diketahui sembarang orang, akhirnya dapat dibaca di internet. Hal ini memungkinkan orang yang tidak dikenali, dapat mengetahui nama lengkap, nama-nama kerabat, sampai alamat rumah yang seharusnya menjadi data pribadi dan tidak diketahui oleh sembarang orang. Semua informasi yang ada di internet merupakan informasi yang sengaja disebarkan atau pun yang secara tidak sadar ikut tersebar di internet.

Kriptografi merupakan salah satu cara yang dapat digunakan untuk menjaga kerahasiaan agar tidak sembarang orang dapat mengetahui suatu informasi. Dalam kriptografi ada dua istilah yang dikenal dengan enkripsi dan dekripsi, yang berarti menyandikan pesan dan memunculkan pesan asli. Proses enkripsi merupakan konversi data ke dalam bentuk lain yang disebut \textit{ciphertext}\footnote{http://searchsecurity.techtarget.com/definition/encryption}. \textit{Ciphertext} tidak dapat dimengerti dengan mudah oleh semua orang, kecuali pihak yang berwenang. Dekripsi merupakan proses pemunculan kembali pesan asli dari \textit{ciphertext}. Sebagian besar \textit{ciphertext} merupakan tulisan yang tidak memiliki arti, sehingga hal ini akan dengan mudah menimbulkan kecurigaan pihak lain bahwa ada suatu informasi yang sebenarnya disembunyikan di dalam \textit{ciphertext} tersebut.

Selain kriptografi, steganografi juga dapat digunakan untuk menjaga kerahasiaan informasi. Steganografi mengatasi permasalahan timbulnya kecurigaan pihak tidak berkepentingan pada saat membaca \textit{ciphertext} yang biasanya merupakan kumpulan huruf dan angka yang tidak memiliki arti. Untuk orang awam yang melihat \textit{ciphertext} mungkin tidak berarti apa-apa. Berbeda halnya jika yang melihat adalah orang yang mengerti kriptografi, dengan sendirinya akan diketahui bahwa ada informasi yang sengaja disembunyikan. Steganografi lebih berfokus pada penyembunyian informasi/pesan rahasia. Dengan menyembunyikan pesan rahasia di dalam pesan lain yang tidak bersifat rahasia, kecurigaan dari pihak tidak berkepentingan pun dapat diatasi.

Ada banyak alasan mengapa steganografi dibutuhkan. Orang-orang yang ingin konseling tentang masalah yang sangat pribadi akan merasa aman. Polisi dapat berkomunikasi dengan \textit{undercover agents} untuk menyergap komplotan orang jahat. Informasi pribadi terjaga dari eksploitasi teroris dan orang jahat.\cite{Dpcrypto:2009}

Steganografi dapat menggunakan berbagai media sebagai media penyembunyian atau yang dikenal dengan istilah \textit{stego-cover}, seperti contohnya media berupa gambar, video, suara, atau teks. Kebanyakan orang memakai gambar sebagai media, karena lebih mudah dalam teknik penyembunyiannya. Media teks masih belum banyak penggunaannya, karena harus memperhatikan cara penulisan, arti dari kalimat, dan lain-lain. Walaupun masih belum banyak yang menggunakannya, media teks merupakan media yang baik untuk menyembunyikan pesan karena kapasitasnya yang cukup besar. Ada berbagai cara menyembunyikan pesan di dalam teks, termasuk memanfaatkan bentuk dari teks itu sendiri (\textit{format-based}) dan memanfaatkan susunan kalimat (\textit{syntax-based}), tetapi belum banyak yang memanfaatkan suku kata.

\section{Rumusan Masalah}
\label{rumusanMasalah}
Berikut beberapa rumusan masalah yang akan dibahas:
\begin{itemize}
	\item Algoritma penyisipan mana yang tepat untuk diimplementasikan dengan memanfaatkan suku kata?
	\item Bagaimana cara implementasi algoritma pada perangkat lunak?
	\item Bagaimana performa yang dihasilkan perangkat lunak?
\end{itemize}

\section{Tujuan}
\label{sec:tujuan}
Berikut merupakan tujuan dari topik skripsi yang saya pilih:
\begin{itemize}
	\item Menemukan algoritma yang mencakup seluruh/hampir seluruh kemungkinan (optimal).
	\item Menentukan seperti apa gambaran perangkat lunak yang akan dibuat.
	\item Mendapatkan performa yang baik dari perangkat lunak.
\end{itemize}

\section{Deskripsi Perangkat Lunak}
\label{sec:Deskripsi Perangkat Lunak}
Perangkat lunak akhir yang akan dibuat memiliki fitur minimal sebagai berikut:
\begin{itemize}
	\item Pengguna dapat menyembunyikan pesan rahasianya ke dalam pesan lain yang tidak bersifat rahasia.
	\item Perangkat lunak akan menghasilkan \textit{stego-object} setelah selesai melakukan proses penyembunyian.
	\item Perangkat lunak dapat mengekstraksi pesan rahasia yang ada di dalam \textit{stego-object}.
\end{itemize}

\section{Metode Penelitian}
\label{sec:metodePenelitian}
Berikut adalah metode penelitian yang digunakan dalam pembuatan skripsi ini:
	\begin{enumerate}
		\item Melakukan studi literatur tentang pengolahan teks.
		\item Melakukan studi literatur tentang pemenggalan kata pada Bahasa Indonesia dan jenis-jenisnya.
		\item Mengumpulkan teks Bahasa Indonesia untuk dijadikan corpus.
		\item Menganalisis kemunculan pola suku kata dari dokumen corpus.
		\item Merancang algoritma pembangkitan cover untuk pesan rahasia.
		\item Mengimplementasikan rancangan algoritma.
		\item Melakukan pengujian.
		\item Membuat dokumentasi skripsi.
	\end{enumerate}