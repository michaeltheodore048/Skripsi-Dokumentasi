\chapter{Kesimpulan dan Saran}
\section{Kesimpulan}
Dari pembuatan perangkat lunak untuk menyembunyikan pesan rahasia dengan menggunakan steganografi berbasis suku kata bahasa Indonesia didapatkan kesimpulan sebagai berikut:
\begin{enumerate}
	\item Algoritma penyisipan dengan memanfaatkan \textit{c}(\textit{w}) dan kamus sinonim memungkinkan suatu \textit{stego-cover} dapat dipakai untuk segala pesan rahasia dengan syarat, kapasitas \textit{stego-cover} dapat menampung pesan rahasia.
	\item Implementasi dilakukan pada bahasa pemrograman Java. Dengan menggunakan bantuan \textit{File Reader}, perangkat lunak dapat mencari sinonim dari kata yang dibutuhkan.
	\item Performa yang dihasilkan perangkat lunak termasuk baik, karena pesan rahasia memang seharusnya tidak terlalu panjang.
	\item Fungsi pemenggalan kata pada perangkat lunak ini memang tidak sempurna, namun hal ini tidak dilihat sebagai kekurangan. Hal ini justru dapat mempersulit penyerang untuk mendapatkan pesan rahasia.
	\item Optimasi terhadap kamus sinonim dapat menurunkan frekuensi kemunculan kata \textit{typo} pada \textit{stego-object} secara signifikan. Hal ini dapat menurunkan tingkat kecurigaan pihak yang tidak bersangkutan saat membaca \textit{stego-object}.
	
\end{enumerate}

\section{Saran}
Saran yang dapat diberikan oleh penulis adalah:
\begin{enumerate}
	\item Memperbanyak karakter yang dapat disembunyikan oleh perangkat lunak.
	\item Mengurutkan penyimpanan kata pada kamus sinonim berdasarkan abjad, untuk mengoptimalkan waktu proses pencarian sinonim.
	\item Memperbaiki algoritma agar pola karakter spasi lebih sulit ditebak.
\end{enumerate}