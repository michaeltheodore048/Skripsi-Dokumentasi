%_____________________________________________________________________________
%=============================================================================
% data.tex v6 (13-04-2015) \ldots dibuat oleh Lionov - Informatika FTIS UNPAR
%
% Perubahan pada versi 6 (13-04-2015)
% - Perubahan untuk data-data ``template" menjadi lebih generik dan menggunakan
%	tanda << dan >>
%
% Perubahan pada versi sebelumnya
% 	versi 5 (10-11-2013)
% 	- Perbaikan pada memasukkan bab : tidak perlu menuliskan apapun untuk 
%	  memasukkan seluruh bab (bagian V)
% 	- Perbaikan pada memasukkan lampiran : tidak perlu menuliskan apapun untuk
%	  memasukkan seluruh lampiran atau -1 jika tidak memasukkan apapun
%	versi 4 (21-10-2012)
%	- Data dosen dipindah ke dosen.tex agar jika ada perubahan/update data dosen
%   mahasiswa tidak perlu mengubah data.tex
%	- Perubahan pada keterangan dosen	
% 	versi 3 (06-08-2012)
% 	- Perubahan pada beberapa keterangan 
% 	versi 2 (09-07-2012):
% 	- Menambahkan data judul dalam bahasa inggris
% 	- Membuat bagian khusus untuk judul (bagian VIII)
% 	- Perbaikan pada gelar dosen
%_____________________________________________________________________________
%=============================================================================
% 								BAGIAN -
%=============================================================================
% Ini adalah file data (data.tex)
% Masukkan ke dalam file ini, data-data yang diperlukan oleh template ini
% Cara memasukkan data dijelaskan di setiap bagian
% Data yang WAJIB dan HARUS diisi dengan baik dan benar adalah SELURUHNYA !!
% Hilangkan tanda << dan >> jika anda menemukannya
%=============================================================================
%_____________________________________________________________________________
%=============================================================================
% 								BAGIAN I
%=============================================================================
% Tambahkan package2 lain yang anda butuhkan di sini
%=============================================================================
\usepackage{booktabs}
\usepackage[table]{xcolor}
\usepackage{longtable}
\usepackage{amsmath}
%=============================================================================

%_____________________________________________________________________________
%=============================================================================
% 								BAGIAN II
%=============================================================================
% Mode dokumen: menetukan halaman depan dari dokumen, apakah harus mengandung 
% prakata/pernyataan/abstrak dll (termasuk daftar gambar/tabel/isi) ?
% - kosong : tidak ada halaman depan sama sekali (untuk dokumen yang 
%            dipergunakan pada proses bimbingan)
% - cover : cover saja tanpa daftar isi, gambar dan tabel
% - sidang : cover, daftar isi, gambar, tabel (IT: UTS-UAS Seminar 
%			 dan UTS TA)
% - sidang_akhir : mode sidang + abstrak + abstract
% - final : seluruh halaman awal dokumen (untuk cetak final)
% Jika tidak ingin mencetak daftar tabel/gambar (misalkan karena tidak ada 
% isinya), edit manual di baris 439 dan 440 pada file main.tex
%=============================================================================
% \mode{kosong}
% \mode{cover}
% \mode{sidang}
%\mode{sidang_akhir}
\mode{final} 
%=============================================================================

%_____________________________________________________________________________
%=============================================================================
% 								BAGIAN III
%=============================================================================
% Line numbering: penomoran setiap baris, otomatis di-reset setiap berganti
% halaman
% - yes: setiap baris diberi nomor
% - no : baris tidak diberi nomor, otomatis untuk mode final
%=============================================================================
\linenumber{yes}
%=============================================================================

%_____________________________________________________________________________
%=============================================================================
% 								BAGIAN IV
%=============================================================================
% Linespacing: jarak antara baris 
% - single: opsi yang disediakan untuk bimbingan, jika pembimbing tidak
%            keberatan (untuk menghemat kertas)
% - onehalf: default dan wajib (dan otomatis) jika ingin mencetak dokumen
%            final/untuk sidang.
% - double : jarak yang lebih lebar lagi, jika pembimbing berniat memberi 
%            catatan yg banyak di antara baris (dianjurkan untuk bimbingan)
%=============================================================================
\linespacing{single}
% \linespacing{onehalf}
%\linespacing{double}
%=============================================================================

%_____________________________________________________________________________
%=============================================================================
% 								BAGIAN V
%=============================================================================
% Bab yang akan dicetak: isi dengan angka 1,2,3 s.d 9, sehingga bisa digunakan
% untuk mencetak hanya 1 atau beberapa bab saja
% Jika lebih dari 1 bab, pisahkan dengan ',', bab akan dicetak terurut sesuai 
% urutan bab.
% Untuk mencetak seluruh bab, kosongkan parameter (i.e. \bab{ })  
% Catatan: Jika ingin menambahkan bab ke-10 dan seterusnya, harus dilakukan 
% secara manual
%=============================================================================
\bab{ }
%=============================================================================

%_____________________________________________________________________________
%=============================================================================
% 								BAGIAN VI
%=============================================================================
% Lampiran yang akan dicetak: isi dengan huruf A,B,C s.d I, sehingga bisa 
% digunakan untuk mencetak hanya 1 atau beberapa lampiran saja
% Jika lebih dari 1 lampiran, pisahkan dengan ',', lampiran akan dicetak 
% terurut sesuai urutan lampiran
% Jika tidak ingin mencetak lampiran apapun, isi dengan -1 (i.e. \lampiran{-1})
% Untuk mencetak seluruh mapiran, kosongkan parameter (i.e. \lampiran{ })  
% Catatan: Jika ingin menambahkan lampiran ke-J dan seterusnya, harus 
% dilakukan secara manual
%=============================================================================
\lampiran{ }
%=============================================================================

%_____________________________________________________________________________
%=============================================================================
% 								BAGIAN VII
%=============================================================================
% Data diri dan skripsi/tugas akhir
% - namanpm: Nama dan NPM anda, penggunaan huruf besar untuk nama harus benar
%			 dan gunakan 10 digit npm UNPAR, PASTIKAN BAHWA BENAR !!!
%			 (e.g. \namanpm{Jane Doe}{1992710001}
% - judul : Dalam bahasa Indonesia, perhatikan penggunaan huruf besar, judul
%			tidak menggunakan huruf besar seluruhnya !!! 
% - tanggal : isi dengan {tangga}{bulan}{tahun} dalam angka numerik, jangan 
%			  menuliskan kata (e.g. AGUSTUS) dalam isian bulan
%			  Tanggal ini adalah tanggal dimana anda akan melaksanakan sidang 
%			  ujian akhir skripsi/tugas akhir
% - pembimbing: isi dengan pembimbing anda, lihat daftar dosen di file dosen.tex
%				jika pembimbing hanya 1, kosongkan parameter kedua 
%				(e.g. \pembimbing{\JND}{  } ) , \JND adalah kode dosen
% - penguji : isi dengan para penguji anda, lihat daftar dosen di file dosen.tex
%				(e.g. \penguji{\JHD}{\JCD} ) , \JND dan \JCD adalah kode dosen
%
%=============================================================================
\namanpm{Michael Theodore Pangestu}{2012730048}	%hilangkan tanda << & >>
\tanggal{<<tanggal>>}{<<bulan>>}{2016}			%hilangkan tanda << & >>
\pembimbing{\TAB}{}     
%Lihat singkatan pembimbing anda di file dosen.tex, hilangkan tanda << & >>
\penguji{<<penguji 1>>}{<<penguji 2>>} 		
%Lihat singkatan penguji anda di file dosen.tex, hilangkan tanda << & >>
%=============================================================================

%_____________________________________________________________________________
%=============================================================================
% 								BAGIAN VIII
%=============================================================================
% Judul dan title : judul bhs indonesia dan inggris
% - judulINA: judul dalam bahasa indonesia
% - judulENG: title in english
% PERHATIAN: - langsung mulai setelah '{' awal, jangan mulai menulis di baris 
%			   bawahnya
%			 - Gunakan \texorpdfstring{\\}{} untuk pindah ke baris baru
%			 - Judul TIDAK ditulis dengan menggunakan huruf besar seluruhnya !!
%			 - Gunakan perintah \texorpdfstring{\\}{} untuk baris baru
%=============================================================================

\judulINA{Steganografi Berbasis Suku Kata Bahasa Indonesia}

\judulENG{Bahasa Indonesia Syllable-Based Steganography}

%_____________________________________________________________________________
%=============================================================================
% 								BAGIAN IX
%=============================================================================
% Abstrak dan abstract : abstrak bhs indonesia dan inggris
% - abstrakINA: abstrak bahasa indonesia
% - abstrakENG: abstract in english
% PERHATIAN: langsung mulai setelah '{' awal, jangan mulai menulis di baris 
%			 bawahnya
%=============================================================================

\abstrakINA{
Kini internet merupakan sumber informasi yang luar biasa lengkap. Semua orang dapat mengakses internet dan berselancar mencari informasi apapun di sana. Informasi yang sengaja diunggah maupun yang secara tidak sadar terunggah pun dapat dilihat. Kebanyakan orang tidak menyadari hal ini. Data pribadi seperti alamat rumah, tanggal lahir, nomor telepon pun tidak sulit untuk dicari di internet. Sekarang ini privasi telah menjadi hal yang sulit didapatkan. Pembicaraan dengan orang-orang terdekat dapat dilacak oleh orang-orang yang ahli di bidangnya.

Steganografi menawarkan solusi untuk masalah ini. Steganografi merupakan cara untuk menyembunyikan pesan rahasia ke dalam pesan yang tidak rahasia (\textit{stego-cover}). Sehingga pesan rahasia akan tetap aman walaupun orang lain membacanya. Steganografi dapat memanfaatkan media seperti gambar, suara, vidio, ataupun teks. Untuk steganografi yang menggunakan teks sebagai medianya juga dibagi lagi berdasarkan bagaimana cara menyembunyikan pesan rahasianya. Ada yang menggeser posisi barisnya, melebarkan spasi, dan memanfaatkan suku kata. Pada skripsi ini akan dibuat perangkat lunak yang memanfaatkan suku kata untuk menyembunyikan pesan rahasia. Setiap kata pada \textit{stego-cover} akan merepresentasikan angka 0 atau 1, bergantung pada banyaknya suku kata pada kata tersebut (ganjil berarti 1 dan genap berarti 0). Notasi \textit{c}(\textit{w}) adalah banyaknya suku kata pada kata \textit{w}. Sedangkan semua karakter yang ada pada pesan rahasia akan diubah ke bentuk ASCII biner. Dengan menyesuaikan \textit{c}(\textit{w}) pada tiap kata di \textit{stego-cover} dengan angka biner pesan rahasia, penyembunyian dapat dilakukan. Tentu saja ada kemungkinan bahwa nilai \textit{c}(\textit{w}) tidak sama dengan angka biner pesan rahasia. Untuk mengatasi hal tersebut, kata akan diganti dengan sinonim yang memiliki \textit{c}(\textit{w}) yang sesuai dengan angka biner pesan rahasia.

Pada skripsi ini berhasil dibuat perangkat lunak yang dapat menyembunyikan dan mengekstrasi kembali pesan rahasia dengan memanfaatkan suku kata. Perangkat lunak ini juga memungkinkan pengguna untuk menambahkan \textit{stego-cover} dengan syarat bahwa pengguna harus melengkapi sinonim untuk kata yang belum memiliki sinonimnya.
}



\abstrakENG{
These days the internet is an incredible source of information. Everyone can access the Internet and surf to find any information there. The information uploaded intentionally or unconsciously uploaded can be seen. Most people do not realize this. Personal data such as home address, date of birth, telephone number was not difficult to find on the internet. Now privacy have become more difficult. Conversation with closest friends and family can be tracked by those skilled.

Steganography offers a solution to this problem. Steganography is a way to hide secret messages into messages that are not confidential (stego-cover). So the secret message will remain secure even if someone else read it. Steganography can use the media like images, sounds, video, or text. Steganography using text as the media are also divided by how to hide secret messages. There are shifted its line, widen the space, and take advantage of syllables. In this thesis will be created software that takes advantage of syllables to hide secret messages. Any word on stego-cover will represent the numbers 0 or 1, depending on the number of syllables in the word (odd and even number means 1 means 0). The notation \textit{c} (\textit{w}) is the number of syllables in the word \textit{w}. While all the characters in the secret message will be converted to ASCII binary form. By adjusting the \textit{c}(\textit {w}) for each word in the stego-cover with binary digit secret message, embedding can be done. Of course there is a possibility that the value of \textit{c}(\ textit{w}) is not equal to binary digit secret message. To overcome this, the word will be replaced with a synonym that has \textit{c}(\textit{w}) corresponding to a binary digit secret message.

In this thesis successfully created software that can embed and extract the secret messages by using syllables. The software also allows users to add more stego-cover on condition that the user must complete synonyms for words that do not have their synonyms.
} 

%=============================================================================

%_____________________________________________________________________________
%=============================================================================
% 								BAGIAN X
%=============================================================================
% Kata-kata kunci dan keywords : diletakkan di bawah abstrak (ina dan eng)
% - kunciINA: kata-kata kunci dalam bahasa indonesia
% - kunciENG: keywords in english
%=============================================================================
\kunciINA{Steganografi, Suku Kata, Steganografi Linguistik}

\kunciENG{Steganography, Syllables, Linguistic Steganography}
%=============================================================================

%_____________________________________________________________________________
%=============================================================================
% 								BAGIAN XI
%=============================================================================
% Persembahan : kepada siapa anda mempersembahkan skripsi ini ...
%=============================================================================
\untuk{<<kepada siapa anda mempersembahkan skripsi ini\ldots?>>}
%=============================================================================

%_____________________________________________________________________________
%=============================================================================
% 								BAGIAN XII
%=============================================================================
% Kata Pengantar: tempat anda menuliskan kata pengantar dan ucapan terima 
% kasih kepada yang telah membantu anda bla bla bla ....  
%=============================================================================
\prakata{\lipsum[3]}
%=============================================================================

%_____________________________________________________________________________
%=============================================================================
% 								BAGIAN XIII
%=============================================================================
% Tambahkan hyphen (pemenggalan kata) yang anda butuhkan di sini 
%=============================================================================
\hyphenation{ma-te-ma-ti-ka}
\hyphenation{fi-si-ka}
\hyphenation{tek-nik}
\hyphenation{in-for-ma-ti-ka}
%=============================================================================


%=============================================================================
